% ============= setup ============= %
% ======== package ======== %
\documentclass[mathserif]{beamer}
\usepackage{xeCJK}
\usepackage{graphicx}
\usepackage{xcolor}
\usepackage{setspace}
\usepackage{newtxmath}

% ======== font ======== %
\setCJKmainfont{Taipei Sans TC Beta}
\setCJKsansfont{Taipei Sans TC Beta}
\AtBeginDocument{%
    \DeclareSymbolFont{pureletters}{OML}{cmm}{m}{it}%
    \SetSymbolFont{pureletters}{bold}{OML}{cmm}{b}{it}%
}
\hypersetup{
    colorlinks=true,
    linkcolor=black,
    urlcolor=blue
}

% ======== theme ======== %
\renewcommand{\baselinestretch}{1.25}
\usetheme{Madrid}
\usecolortheme{crane}
\setbeamertemplate{items}[circle]
\setbeamertemplate{section in toc}{\inserttocsectionnumber.~\inserttocsection}
\AtBeginSection[]{
    \begin{frame}
        \vfill
        \centering
        \begin{beamercolorbox}[sep=8pt,center,shadow=true,rounded=true]{title}
            \usebeamerfont{title}\insertsectionhead\par%
        \end{beamercolorbox}
        \vfill
    \end{frame}
}

% ======== data ======== %
\title{實作能力加強}
\author{temmie}
\date{}

% ============= setup ============= %

\begin{document}

\begin{frame}
    \titlepage
\end{frame}

\begin{frame}
    \tableofcontents
\end{frame}

\section{位元運算}

\begin{frame}
    \frametitle{為什麼要學這個?}
    \begin{itemize}
        \item 因為電腦的機制,位元運算會比一般運算更快一些
        \item 在\textbf{枚舉}這堂課上,我們會用到很多
    \end{itemize}
\end{frame}

\begin{frame}
    \frametitle{進位制}
    \begin{itemize}
        \item n 進位代表只用 n 以內的數字組成(如果超過 10 就用英文)
        \item 二進位的數字代表只用 0 和 1 組成
        \item<2-> 如果一個式子有不同進位制的數字,則會用括弧備註在右下角
        \item<2-> 例如:$17_{(10)}=10001_{(2)}$
    \end{itemize}
\end{frame}

\begin{frame}
    \frametitle{二進位轉十進位}
    \begin{itemize}
        \item 以 $10001_{(2)}$ 舉例
        \item 我們可以先把所有 2 的冪次列出來:$2^4$、$2^3$、$2^2$、$2^1$、$2^0$
        \item<2-> 和原本的二進位字串一一對應後相乘
        $1\times 2^4$、$0\times 2^3$、$0\times 2^2$、$0\times 2^1$、$1\times 2^0$
        \item<3-> 將所有數字相加
        $1\times 2^4+0\times 2^3+0\times 2^2+0\times 2^1+1\times 2^0=17$
    \end{itemize}
\end{frame}

\begin{frame}
    \frametitle{十進位轉二進位}
    \begin{itemize}
        \item 詳細作法如下圖
        \item \includegraphics[width=7.0cm]{img/1-1.png}
    \end{itemize}
\end{frame}

\begin{frame}
    \frametitle{例題}
    \begin{block}{二進位制轉換}
        \href{https://zerojudge.tw/ShowProblem?problemid=a034}{題目連結}\\
        給你一個十進位的正整數 $n$,輸出 $n$ 的二進位
    \end{block}
\end{frame}

\begin{frame}
    \frametitle{位元運算的類別}
    \begin{itemize}
        \item 主要有 OR、AND、XOR、NOT
        \item 和一般的 or、and、not 不同的是,可以對數字做運算,而不是限制在布林
        \vspace{0.5cm}
        \item<2-> $AND$: 「 & 」 表示,如果兩個 bit \textbf{都是} 1 就是 1,否則為 0
        \item<2-> $OR$: 「 | 」 表示,如果兩個 bit \textbf{至少}有一個 1 就是 1,否則為 0
        \item<2-> $XOR$: 「 ^ 」 表示,如果兩個 bit \textbf{恰有}一個為 1,否則為 0
        \item<2-> $NOT$: 「 ~ 」表示,將 1 變成 0,0 變成 1
        % 記得手算真值表舉例
    \end{itemize}
\end{frame}

\begin{frame}
    \frametitle{使用位元運算}
    \begin{itemize}
        \item 難道用位元運算還要手動轉進位制? @@
        \item<2-> 在 C++ 中,直接對兩個十進位的數字做就可以了
        \item<2-> $17_{(10)} \mid 5_{(10)} = 10001_{(2)} \mid 00101_{(2)} = 10101_{(2)} = 21_{(10)}$
    \end{itemize}
\end{frame}

\begin{frame}
    \frametitle{例題}
    \begin{block}{位元運算的大雜燴}
        \href{https://codeforces.com/group/S6XjkGb6qB/contest/403070/problem/B}{題目連結}\\
        給你三個參數 $x$、$y$、$z$ ,找到 $(x \oplus (y \mid z)) \oplus (x+y)$ 的最小值\\
        其中 $\oplus$ 是 XOR 的意思,$\mid$ 是 OR 的意思
    \end{block}
    \begin{itemize}
        \item<2-> 參數數量是固定的,那就列出所有情況吧!
        \item<2-> 把六種組合各試一遍就可以了
    \end{itemize}
\end{frame}

\begin{frame}
    \frametitle{更多位元運算}
    \begin{itemize}
        \item 另外常用的位元運算有左移和右移
        \vspace{0.5cm}
        \item<2-> 左移: 「 << 」 表示,代表將所有 bit 左移 n 位
        \item<2-> 右移: 「 >> 」 表示,代表將所有 bit 右移 n 位
        \vspace{0.5cm}
        \item<3-> 例如: $17_{(10)} << 1 = 10001_{(2)} << 1 = 100010_{(2)} = 34_{(10)}$
        \item<3->你知道嗎?實際上左移就是乘上 $2^n$(右移則相反)
    \end{itemize}
\end{frame}

\section{字元轉換}

\begin{frame}
    \frametitle{Ascii}
    \begin{itemize}
        \item 我們都知道電腦是儲存 0 跟 1,並不能儲存字元
        \item 那要怎麼儲存字元呢?
        \item<2-> 只要把所有字母編碼就可以啦!
        \item<2-> 實際上我們用的字元都有一個編碼,稱為 Ascii 編碼
    \end{itemize}
\end{frame}

\begin{frame}
    \frametitle{Ascii}
    \begin{itemize}
        \item $[\ 0-9\ ]$ = 48 $\sim$ 57
        \item $[\ A-Z\ ]$ = 65 $\sim$ 90
        \item $[\ a-z\ ]$ = 97 $\sim$ 122
        \item 利用編碼的連續性,就有很多功能可以應用
        % 可以用 $ ascii -d 演示
    \end{itemize}
\end{frame}

\begin{frame}
    \frametitle{例題}
    \begin{block}{數字總和}
        \href{https://codeforces.com/group/S6XjkGb6qB/contest/403070/problem/C}{題目連結}\\
        你有一個數字 $n$,請你求出所有位數的總和,保證 $n$ 的位數 $\leq 10^5$
    \end{block}
    \begin{itemize}
        \item<2-> 字串很大,看起來沒辦法用 int 儲存,所以只能用 string 儲存
        \item<2-> 我們可以用 for 得到每個字元,並且\textbf{減去 '0'},就可以轉換成數字
        \item<2-> '0'(48)-'0'(48)=0,'1'(49)-'0'(48)=1
        \item<2-> '5'(53)-'0'(48)=5,'9'(57)-'0'(48)=9
    \end{itemize}
\end{frame}

\begin{frame}
    \frametitle{例題}
    \begin{block}{凱撒密碼}
        \href{https://zerojudge.tw/ShowProblem?problemid=b516}{題目連結}\\
        給你一個字串 $s$,把每個字母向後移 3 位,例如 A → D,Z → C
    \end{block}
    \begin{itemize}
        \item<2-> 我們可以將每個字母的值都加上 3
        \item<2-> 如果該字母超過 'Z' 的話就減去 26
    \end{itemize}
\end{frame}

\section{陣列運用}

\begin{frame}
    \frametitle{陣列運用}
    \begin{itemize}
        \item 陣列的用途很廣,巧妙的使用陣列可以讓時間複雜度更好
        \item 用陣列也可以實做很多好用的資料結構,並提供額外的性質
    \end{itemize}
\end{frame}

\begin{frame}
    \frametitle{vector}
    \begin{itemize}
        \item vector 可以說是更好的陣列,\textbf{可以動態的分配空間}
        \item 陣列做的事 vector 做的到,陣列不能做的事 vector 也可以做的到
        \item 缺點:可能比陣列慢一點點、大量宣告可能會吃到 MLE
    \end{itemize}
\end{frame}

\begin{frame}
    \frametitle{vector 的宣告和使用}
    vector<{\color[rgb]{1,0,0}type}> {\color[rgb]{1,0,0}name}({\color[rgb]{1,0,0}size}, {\color[rgb]{1,0,0}value});
    \begin{itemize}
        \item type 是放 vector 裡面東西的型別
        \item name 則是 vector 的名稱(接下來的介紹都用 vec 表示)
        \item (可選)size 可以放初始的陣列大小
        \item (可選)value 可以放初始值
    \end{itemize}
    \vspace{0.5cm}
    {\color[rgb]{1,0,0}name}.{\color[rgb]{1,0,0}method}({\color[rgb]{1,0,0}value1}, {\color[rgb]{1,0,0}value2}...)
    \begin{itemize}
        \item 以上為 vector 如何使用的模板
    \end{itemize}
\end{frame}


\begin{frame}
    \frametitle{vector 的資料位置}
    \begin{itemize}
        \item .begin() 代表 vector 裡第一個資料位置
        \item .end() 代表 vector 裡最後一個資料\textbf{的後一位}位置
        \item vec[i] 代表 vector 裡的第 i 個資料
        \item<2-> \includegraphics[width=8.0cm]{img/3-1.png}
        \item<2-> 要對整個 vector 操作丟 .begin() 跟 .end() 就好了
    \end{itemize}
\end{frame}

\begin{frame}
    \frametitle{vector 的使用}
    \begin{itemize}
        \item 以後介紹的函式,除非有特別標注,否則都為以下的方法
        \item 開始都是包含,結束則不包含
        \item 書寫的形式為 [ L, R )
        \item \includegraphics[width=8.0cm]{img/3-2.png}
    \end{itemize}
\end{frame}

\begin{frame}
    \frametitle{vector 的輸入方式}
    \begin{itemize}
        \item 分配好空間,並且用 cin >> [i]
        \item 不分配空間,用 vec.push\_back({\color[rgb]{1,0,0}value}) 將元素放入最後一位
        \vspace{0.5cm}
        \item 要用哪一個?如果後續沒有要操作就用第ㄧ個,否則用另一個
    \end{itemize}
\end{frame}

\begin{frame}
    \frametitle{vector 的各種常用操作}
    \begin{itemize}
        \item $O(1)$,.size(),取得 vector 的大小
        \item $O(1)$,.empty(),回傳 vector 是否為空
        \item $O(\mid size-n \mid )$,.resize({\color[rgb]{1,0,0}n}, {\color[rgb]{1,0,0}[val]})\\
        重設 vector 的大小,並設為 val(可選)
        \item $O(1)$,.pop\_back(),清除 vector 最後面的元素
        \item $O(1)$,.clear(),清空 vector
        \item $O(1)$,.front(),回傳最前面的元素
        \item $O(1)$,.back(),回傳最後面的元素
    \end{itemize}
\end{frame}

\begin{frame}
    \frametitle{vector 的遍歷}
    \begin{itemize}
        \item \includegraphics[width=5.0cm]{img/3-3.png}
        \item 其中 x 就會是 vector 裡的所有元素
        \item 想要改值?回想前一節講的吧!
    \end{itemize}
\end{frame}

\begin{frame}
    \frametitle{例題}
    \begin{block}{圖書館}
        \href{https://zerojudge.tw/ShowProblem?problemid=f819}{題目連結}\\
        給你 $n$ 本書,每本書都有編號 $a_i$ 和借閱日期 $b_i$\\
        如果借閱日期超過 100 天就代表逾期,每超過 1 天就要罰 5 元罰金\\
        請求出所有逾期的書的編號,以及總罰金
    \end{block}
    \begin{itemize}
        \item<2-> 我們可以把所有逾期的書都裝進 vector 裡,是不是簡單又方便呢?
    \end{itemize}
\end{frame}

\begin{frame}
    \frametitle{二維 vector 的宣告和使用}
    vector<vector<{\color[rgb]{1,0,0}type}>> {\color[rgb]{1,0,0}name}({\color[rgb]{1,0,0}size1}, vector<int>({\color[rgb]{1,0,0}size2}, {\color[rgb]{1,0,0}value}))
    \begin{itemize}
        \item 以上如同 int[ ][ ]
        \item 不過實在是太長了,通常我還是會用 int[ ][ ]
    \end{itemize}
    \vspace{0.5cm}
    vector<{\color[rgb]{1,0,0}type}> {\color[rgb]{1,0,0}name}[{\color[rgb]{1,0,0}size}]
    \begin{itemize}
        \item 這也是二維陣列,不過每一項都可以是不同大小
        \item 在\textbf{圖論}這個單元我們會很常用到
    \end{itemize}
    \vspace{0.5cm}
    該學哪個?一如往常的,兩個都要學
\end{frame}

\begin{frame}
    \frametitle{二維 vector 的比較}
    \vspace{0.5cm}
    \includegraphics[width=7.0cm]{img/3-4.png}
\end{frame}

\begin{frame}
    \frametitle{例題}
    \begin{block}{vector練習}
        \href{https://toj.tfcis.org/oj/pro/575/}{題目連結}\\
        給你 $n$ 個人,並且有 $k$ 組朋友關係,由小到大輸出每個人的朋友有哪些\\
        每組朋友關係有兩個整數 $a$、$b$,代表 $a$ 和 $b$ \textbf{互相為}朋友\\
        (記得 IO 加速)
    \end{block}
    \begin{itemize}
        \item<2-> 這題其實就是圖的儲存!我們留到圖論在說
    \end{itemize}
\end{frame}

\begin{frame}
    \frametitle{陣列運用 — 計數}
    \begin{itemize}
        \item 要快速獲得一個元素出現的數量,可以直接在陣列儲存該值的數量
        \item 修改元素也可以快速的計算!
    \end{itemize}
\end{frame}

\begin{frame}
    \frametitle{例題}
    \begin{block}{數對問題-1}
        請找出長度為 $n$ 的陣列裡有多少數對 ($arr_i, arr_j$) 的和為 $m$
        \vspace{0.5cm}
        \begin{itemize}
            \item $1 \leq n \leq 10^5$
            \item $0 \leq m \leq 10^5$
            \item $0 \leq arr_i \leq m$
        \end{itemize}
    \end{block}

    \begin{itemize}
        \item<2-> 暴力求解? $O(n^2)$ 會 TLE
        \item<3-> 移項 $arr_i+arr_j=m \rightarrow arr_i=m-arr_j$
        \item<3-> 對於每個 $arr_i$ 只要找 $m-arr_j$ 的數量即可,$O(n)$ 即可得到答案
    \end{itemize}
\end{frame}

\begin{frame}
    \frametitle{例題}
    \begin{block}{Missing Number}
        請找出長度為 $1 \sim n$ 中的 $n-1$ 個數字,請找出少了哪個。
        \vspace{0.5cm}
        \begin{itemize}
            \item $2 \leq n \leq 2 \times 10^5$
        \end{itemize}
    \end{block}

    \begin{itemize}
        \item<2-> 用一個陣列儲存一個數字是否出現過
        \item<2-> 掃過 $1 \sim n$ 即可知道哪個沒有出現過
        \item<3-> 你能想到不用陣列的解法嗎?
    \end{itemize}
\end{frame}

\begin{frame}
    \frametitle{限制}
    \begin{itemize}
        \item 上面的兩個問題數字都很小,都可以用陣列儲存
        \item 如果數字變大呢?以後會有更多方式解決!(map、離散化…)
        \item<2-> 不過\textbf{更大的數字會犧牲掉時間複雜度}
        \item<2-> 如果能用陣列解決就用陣列
    \end{itemize}
\end{frame}

\section{常用 STL 函式}

\begin{frame}
    \frametitle{STL 是什麼?}
    \begin{itemize}
        \item STL 全名為 Standard Template Library
        \item 是由 C++ 提供的一個標準模板庫
        \item 我們比較常使用到 STL 中「函式」、「容器」、「迭代器」等部份
        \vspace{0.5cm}
        \item STL 容器等到\textbf{資料結構}的課再介紹
    \end{itemize}
\end{frame}

\begin{frame}
    \frametitle{常用 STL 函式}
    \begin{itemize}
        \item fill
        \item reverse
        \item swap
        \item min / max
        \item min\_elementm / max\_element
        \item sort
    \end{itemize}
\end{frame}

\begin{frame}
    \frametitle{fill}
    \begin{itemize}
        \item 時間複雜度: \textcolor{red}{\textbf{$O(n)$}}
        \item fill(\textcolor{red}{L}, \textcolor{red}{R}, \textcolor{red}{val})
        \item 把 [L, R) 之間的元素都改成 val
        \vspace{0.5cm}
        \item 主要用來對容器做初始化
        \item 要注意時間複雜度,不要使用太多次
    \end{itemize}
\end{frame}

\begin{frame}
    \frametitle{reverse}
    \begin{itemize}
        \item 時間複雜度: \textcolor{red}{\textbf{$O(n)$}}
        \item reverse(\textcolor{red}{L}, \textcolor{red}{R})
        \item 把 [L, R) 之間的元素反轉
    \end{itemize}
\end{frame}

\begin{frame}
    \frametitle{swap}
    \begin{itemize}
        \item 時間複雜度: \textcolor{red}{\textbf{$O(n)$}}
        \item swap(\textcolor{red}{A}, \textcolor{red}{B})
        \item 把 A 跟 B 裡面的內容對調
    \end{itemize}
\end{frame}

\begin{frame}
    \frametitle{min / max}
    \begin{itemize}
        \item 時間複雜度: $O(1)$
        \item min(\textcolor{red}{A}, \textcolor{red}{B})
        \item 回傳 A 跟 B 哪個比較大 / 小
    \end{itemize}
\end{frame}

\begin{frame}
    \frametitle{min / max(多個元素)}
    \begin{itemize}
        \item 最大值跟最小值都有結合律
        \item 寫法ㄧ:min(A, min(B, C))
        \item 寫法二:min(\{A, B, C\}) (只支援 C14 以上)
    \end{itemize}
\end{frame}

\begin{frame}
    \frametitle{min\_elementm / max\_element}
    \begin{itemize}
        \item 時間複雜度: \textcolor{red}{\textbf{$O(n)$}}
        \item min\_elementm(\textcolor{red}{L}, \textcolor{red}{R}, \textcolor{red}{val})
        \item 回傳 [L, R) 區間的最大值 / 最小值\textbf{的位置}
        \vspace{0.5cm}
        \item 如果想要取得實際的值要在前面加上「\textcolor{red}{\textbf{*}}」,如下
        \item *min\_element(v.begin(), v.end())
    \end{itemize}
\end{frame}

\begin{frame}
    \frametitle{sort}
    \begin{itemize}
        \item 時間複雜度: \textcolor{red}{\textbf{$O(n\ log\ (n))$}}
        \item sort(\textcolor{red}{L}, \textcolor{red}{R}, \textcolor{red}{val})
        \item 把 [L, R) 之間的元素由小到大排序
        \vspace{0.5cm}
        \item 對區間做排序,這個函式非常常用到!
        \item 通常排序後的東西會有特殊的性質,這個我們以後再講
    \end{itemize}
\end{frame}

\begin{frame}
    \frametitle{sort 自訂排序}
    \begin{itemize}
        \item 上面提到 sort 是從小到大排,不過我們有沒有辦法從大到小呢?
        \item 實際上,只要告訴程式一個定義好的比較方式就可以了!
        \item 例如:如果 a 元素比較大,就放在前面,否則放在後面
    \end{itemize}
\end{frame}

\begin{frame}
    \frametitle{sort 自訂排序}
    \begin{itemize}
        \item 實作方式如下
        \item \includegraphics[width=9.0cm]{img/4-1.png}
    \end{itemize}
\end{frame}

\section{包裝元素}

\begin{frame}
    \frametitle{為什麼要包裝元素}
    \begin{itemize}
        \item 讓我們假想一個問題,教室裡有很多學生,要分別紀錄他們的「名稱」、「座號」跟「各科成績」,該怎麼做?
        \item<2-> 聽起來很簡單,用三個陣列存資訊就好
        \item<3-> 如果我需要變動一位學生的順序呢?三個陣列都要修改,好麻煩
        \item<4-> 把他們打包吧!放在一起就很方便了
    \end{itemize}
\end{frame}

\begin{frame}
    \frametitle{struct}
    \begin{itemize}
        \item struct 是一個可以打包元素並創建一個自訂的型別
        \item 讓我們看看下面的語法範例:
        \item \includegraphics[width=7.0cm]{img/5-1.png}
        \item<2-> 我們宣告了一個型別叫做「student」,並且可以紀錄他的名稱,座號跟各科成績
        \item<2-> 此外還宣告了一個陣列儲存多個學生的資訊。
    \end{itemize}
\end{frame}

\begin{frame}
    \frametitle{struct}
    \begin{itemize}
        \item 好耶,我們現在有一個自己的型別了,不過有那麼多資料,要怎麼存取呢?
        \item 可以使用「\textbf{\textcolor{red}{.}}」存取資料
        \item<2-> 例如:「students[26]」代表「陣列索引值為 26 的同學的所有資訊」
        \item<2-> 「students[17].id」代表「陣列索引值為 17 的同學的座號」
    \end{itemize}
\end{frame}

\begin{frame}
    \frametitle{例題}
    \begin{block}{圖書館小志工}
        \href{}{題目連結}\\
        目前書架上有 $n$ 本書,分別叫做 $a_1, a_2 \cdots a_n$\\
        每本書紀錄著「書名」、「分類」、「編號」\\
        .\\
        現在給你 $m$ 個操作,每次操作會把 $a_x$ 跟 $a_y$ 交換\\
        請求出書架最後的狀態
    \end{block}
\end{frame}

\section{區間問題}

\begin{frame}
    \frametitle{區間問題概述}
    \begin{itemize}
        \item 在競程裡面,我們有相當多的問題是有關於「區間」的
        \item 通常會有兩個功能「修改」跟「查詢」
        \item 這次我們會介紹兩個非常常見且基礎的方法
    \end{itemize}
\end{frame}

\begin{frame}
    \frametitle{例題}
    接下來我們用幾個問題來暖身

    \begin{itemize}
        \item<1-> 1+3+2+1+3+2+1+3+3+2+1+2+3
        \item<2-> 1+3+2+1+3+2+1+3+3+2+1+2+3\textcolor{red}{+5}
        \item<3-> 1+3+2+1+3+2+1+3+3+2+1+2+3\textcolor{red}{+5+3}
        \vspace{0.5cm}
        \item<4-> 4+1+3+\textcolor{blue}{1+3+3+4+3+1+2+1+2}+1
        \item<5-> 4+\textcolor{blue}{1+3+1+3+3+4+3+1+2}+1+2+1
        \vspace{0.5cm}
        \item<6-> 如果一個一個計算是不是很慢呢?讓我們透過一些技巧加速
    \end{itemize}
\end{frame}

\begin{frame}
    \frametitle{前綴和概念}
    \begin{itemize}
        \item 前綴和可以處理\textbf{沒有修改,區間查詢}的問題
        \item 其中每個查詢都可以在 $O(1)$ 完成
        \item 非常非常非常重要的技巧,很常出現
    \end{itemize}
\end{frame}

\begin{frame}
    \frametitle{前綴和概念}
    \begin{itemize}
        \item \includegraphics[width=3.0cm]{img/6-1.png}
        \item 上圖的藍色區域怎麼算呢?
        \item<2-> 只要算出正方形的面積,並且減去扇形,即可求解
        \item<2-> 讓我們看看這個概念放到陣列上會變成怎樣
    \end{itemize}
\end{frame}

\begin{frame}
    \frametitle{前綴和概念}
    \begin{itemize}
        \item \includegraphics[width=8.0cm]{img/6-2.png}
        \item 我們可以算出所有數字的總和,然後扣除灰色的部份
        \item<2-> 嗯?這樣應該算的數字更多,因此更慢不是嗎?
        \item<2-> 實際上,我們可以儲存這些數字,要用時再扣除,這樣就很快啦!
    \end{itemize}
\end{frame}

\begin{frame}
    \frametitle{前綴和概念}
    \begin{itemize}
        \item \includegraphics[width=4.0cm]{img/6-3.png}
        \item 試著把上面的藍色數值兩兩相減看看,然後觀察他們的性質吧
        \item 如果我想要獲得一個區間的和,該怎麼求得呢?
        \item 所有藍色的值需要一個一個計算嗎?還是有更快的方法?
    \end{itemize}
\end{frame}

\begin{frame}
    \frametitle{前綴和概念}
    \begin{itemize}
        \item 稍微整理一下,要找到一個區間 [ L, R ] 的和
        \item 則將 [ 0, R ] - [ 0, L-1 ] 即可
        \item 而 [ 0, n ] 的值可以從 [ 0, n-1 ] + $a_n$ 得到
    \end{itemize}
\end{frame}

\begin{frame}
    \frametitle{前綴和實做}
    \begin{itemize}
        \item \includegraphics[width=9.0cm]{img/6-4.png}
        \item 上面這個過程叫做預處理,可以看出時間複雜度為 $O(n)$
        \item 我們定義 $pre_i$ 為 $\displaystyle\sum_{j=0}^{i} a_j$
    \end{itemize}
\end{frame}

\begin{frame}
    \frametitle{例題}
    \begin{block}{前綴和陣列實作}
        \href{https://zerojudge.tw/ShowProblem?problemid=e339}{題目連結}\\
        如題,試著實作出我們剛剛介紹的前綴和陣列
    \end{block}
\end{frame}

\begin{frame}
    \frametitle{前綴和實做}
    \begin{itemize}
        \item \includegraphics[width=8.0cm]{img/6-5.png}
        \item 每次查詢都直接拿陣列裡面的值,時間複雜度為 $O(q)$
    \end{itemize}
\end{frame}

\begin{frame}
    \frametitle{例題}
    \begin{block}{一維區間和問題}
        \href{https://cses.fi/problemset/task/1646}{題目連結}\\
        給你一個有 $n$ 個正整數的陣列 $arr$\\
        給予 $q$ 個查詢,每個查詢都有兩個正整數 $L_i$,$R_i$\\
        對於每個查詢求出 [ $L_i$, $R_i$ ] 的總和\\

        保證 $1 \leq n, q \leq 2 \times 10^5$
    \end{block}
\end{frame}

\begin{frame}
    \frametitle{例題}
    \begin{block}{二維區間和問題}
        \href{https://cses.fi/problemset/task/1652}{題目連結}\\
        給你一個 $n \times n$ 大小的森林,森林的每一個座標都是空地或是樹\\
        給予 $q$ 個查詢,每個查詢包含 $y_{1_i}$、$x_{1_i}$、$y_{2_i}$、$x_{2_i}$
        對於每個查詢輸出該範圍有多少顆樹
    \end{block}
    \begin{itemize}
        \item 請你想想看,如何運用同樣的概念做二維的區間查詢呢?
        \item<2-> 我們一樣儲存 $(0, 0)$ 到 $(x, y)$ 的區間和
        \item<2-> 並用相同的\textbf{扣除不需要的範圍}的概念
    \end{itemize}
\end{frame}

\begin{frame}
    \frametitle{例題}
    \begin{itemize}
        \item \includegraphics[width=6.0cm]{img/6-6.png}
    \end{itemize}
\end{frame}

\begin{frame}
    \frametitle{例題}
    \begin{itemize}
        \item \includegraphics[width=6.0cm]{img/6-7.png}
        \item 先把最大的問題加起來
    \end{itemize}
\end{frame}

\begin{frame}
    \frametitle{例題}
    \begin{itemize}
        \item \includegraphics[width=6.0cm]{img/6-8.png}
        \item 將不需要的範圍刪除
    \end{itemize}
\end{frame}

\begin{frame}
    \frametitle{例題}
    \begin{itemize}
        \item \includegraphics[width=6.0cm]{img/6-9.png}
        \item 把多刪除的範圍加回去
    \end{itemize}
\end{frame}

\begin{frame}
    \frametitle{例題}
    \begin{block}{區間 xor 問題}
        \href{https://cses.fi/problemset/task/1650}{題目連結}\\
        給你一個有 $n$ 個正整數的陣列 $arr$\\
        給予 $q$ 個查詢,每個查詢都有兩個正整數 $L_i$,$R_i$\\
        對於每個查詢求出 [ $L_i$, $R_i$ ] 所有值做 xor 後的值\\

        保證 $1 \leq n, q \leq 2 \times 10^5$
    \end{block}
    \begin{itemize}
        \item 仍然是使用相同的概念,不過 xor 的值要怎麼消除呢?
        \item tip: $A \oplus A = 0$
        \item<2-> 我們可以可以做出類似前綴和陣列的東西,不過加的功能變成 xor
        \item<2-> 在查詢時,也使用 xor 消除
    \end{itemize}
\end{frame}

\begin{frame}
    \frametitle{例題}
    \includegraphics[width=10.0cm]{img/6-10.png}
\end{frame}

\begin{frame}
    \frametitle{差分概念}
    \begin{itemize}
        \item 還記得我們說過前綴和的使用時機嗎?
        \item 讓我們把問題倒過來,有沒有辦法做到\textbf{區間修改,沒有查詢}
        \item 比起前綴和,差分的概念較為抽象
    \end{itemize}
\end{frame}

\begin{frame}
    \frametitle{差分概念}
    \begin{itemize}
        \item \includegraphics[width=10.0cm]{img/6-11.png}
        \item 在差分裡,我們會儲存跟前面數字的差,以方便修改
    \end{itemize}
\end{frame}

\begin{frame}
    \frametitle{差分概念}
    \begin{itemize}
        \item \includegraphics[width=10.0cm]{img/6-12.png}
        \item 可以透過圖示發現,修改區間的值實際上只會動到兩格
        \item 因為區間內的值的變化量一樣,所以差一樣($a-b=(a+x)-(b+x)$)
    \end{itemize}
\end{frame}

\begin{frame}
    \frametitle{差分概念}
    \begin{itemize}
        \item 稍微整理一下,我們會有初始的差分序列
        \item 每次修改就是對 L 加上 value,而 R+1 加上 -value
        \item 如果需要尋找一個點的值,則要從左到又加上所有變化量
    \end{itemize}
\end{frame}

\begin{frame}
    \frametitle{例題}
    \begin{block}{差分陣列實作}
        \href{https://zerojudge.tw/ShowProblem?problemid=e340}{題目連結}\\
        如題,試著實做出我們剛剛介紹的前綴和陣列
    \end{block}
\end{frame}

\begin{frame}
    \frametitle{例題}
    \begin{block}{疊鬆餅}
        \href{https://codeforces.com/problemset/problem/1501/B}{題目連結}\\
        你有 $n$ 個鬆餅需要從底層疊到頂層,每個鬆餅都有一個溼潤度 $a_i$,代表將該鬆餅放置頂層後,會讓底下 $a_i$ 個(包含自己)鬆餅都沾上奶油
        請你輸出疊完這 $n$ 個鬆餅後哪些鬆餅沾上了奶油\\
         \\
        保證 $1 \leq n \leq 2 \times 10^5$ 且 $0 \leq a_i \leq n$
    \end{block}
\end{frame}

\end{document}