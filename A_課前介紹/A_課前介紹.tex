\documentclass{beamer}
\usepackage{xeCJK}
\usepackage{graphicx}
\usepackage{xcolor}

% ============= setup =============
\usetheme{Madrid}
\usecolortheme{crane}
\setCJKmainfont{Taipei Sans TC Beta}
\setmainfont{Taipei Sans TC Beta}
\hypersetup{
    colorlinks=true,
    linkcolor=black,
    urlcolor=blue
}
\setbeamertemplate{items}[circle]
\setbeamertemplate{section in toc}{\inserttocsectionnumber.~\inserttocsection}
\AtBeginSection[]{
    \begin{frame}
    \vfill
    \centering
    \begin{beamercolorbox}[sep=8pt,center,shadow=true,rounded=true]{title}
        \usebeamerfont{title}\insertsectionhead\par%
    \end{beamercolorbox}
    \vfill
    \end{frame}
}

\title{課前介紹}
\author{temmie}
\date{}
% ============= setup =============

\begin{document}

\begin{frame}
    \titlepage
\end{frame}

\begin{frame}
    \tableofcontents
\end{frame}

\section{課程相關}

\begin{frame}
    \frametitle{課程規範}
    \begin{itemize}
        \item 課程中禁止玩手機遊戲、線上遊戲
        \item 需達到 60\% 以上的出席率
        \item 不會特別去檢查作業的完成度,但是每個主題請盡量寫過一些題目
    \end{itemize}
\end{frame}

\begin{frame}
    \frametitle{攜帶物品}
    \begin{itemize}
        \item 白紙、文具
        \item 需要的話可以自己帶筆電
    \end{itemize}
\end{frame}

\begin{frame}
    \frametitle{課程設計}
    \begin{itemize}
        \item 主要是讓各位在每週都有固定的進度可以上
        \item 而實做、練習的部份就是放在課後
    \end{itemize}
\end{frame}

\section{學習技巧}

\begin{frame}
    \frametitle{如何做筆記}
    \begin{itemize}
        \item \href{https://hackmd.io/?nav=overview}{Hackmd — 線上筆記軟體}
        \item \href{https://youtu.be/Egj_DdGUIDI}{Obsidian — 線下卡片盒筆記}
        \vspace{0.5cm}
        \item 需要的話,你可以自己帶筆電來
    \end{itemize}
\end{frame}

\begin{frame}
    \frametitle{做筆記的原則}
    \begin{itemize}
        \item 大原則:確保自己在一個禮拜後還能看得懂筆記
        \item 如果有 code 可以加上註解並且定義每個參數的意義
        \item 做好分類:以類似題目類型(搜尋、圖論、資料結構…)整理
        \vspace{0.5cm}
        \item 這是我寫的「\href{https://hackmd.io/@temmie950807/rJL9TfXpj}{用 Obsidian 寫競程筆記}」
    \end{itemize}
\end{frame}

\begin{frame}
    \frametitle{如何問問題}
    \href{https://github.com/ryanhanwu/How-To-Ask-Questions-The-Smart-Way}{參考網址}
    
    \vspace{0.5cm}
    或是遵守以下原則
    \begin{itemize}
        \item 詳細描述問題
        \item 傳題目網址
        \item 傳程式碼
        \item 有錯誤給錯誤訊息
    \end{itemize}
\end{frame}

\begin{frame}
    \frametitle{如何學習}
    \begin{itemize}
        \item 多刷題(有效率的刷)
        \item 到\href{https://discord.gg/cisc}{社群}上面討論
        \item 打完比賽後補題
    \end{itemize}
\end{frame}

\section{課程主題}

\begin{frame}
    \frametitle{主題概念}
    \begin{itemize}
        \item 資料結構
        \item 枚舉
        \item 貪心
        \item 搜尋
        \item 分治
        \item 圖論
        \item 動態規劃
        \item 字串
        \item 計算幾何
    \end{itemize}

    \vspace{0.5cm}
\end{frame}

\end{document}