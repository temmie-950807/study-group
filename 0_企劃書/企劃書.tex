\documentclass[12pt]{article}
\usepackage{xeCJK}
\setCJKmainfont[]{Noto Serif CJK TC}
\usepackage[top=2cm, bottom=2cm, left=3cm, right=3cm]{geometry}
\usepackage{setspace}
\onehalfspacing
\usepackage{xcolor}
\usepackage{graphicx}
\renewcommand{\baselinestretch}{1.5}

\begin{document}
    \begin{Huge}
        \begin{center}
            資訊檢定 × 競程讀書會企劃書
        \end{center}
    \end{Huge}

    \section*{壹、計畫名稱}
    資訊檢定 × 競程讀書會企劃書

    \section*{貳、計畫宗旨}
    目標為對成淵高中校內對程式設計有興趣且有 python 或是 C++ 基礎的高一、二生打造更好的學習環境以及共學夥伴。\\以競賽程式和 APCS 檢定為目標,提供優質的課程以及學習資源。

    \section*{參、計畫內容}
    以每週的固定課程,讓同學們不僅可以在自身的技術上進步,也可以和其他同學交流,了解不同的解題思路。

    \section*{肆、地點}
    圖書館檢索區

    \section*{伍、參與人員}
    幹部:
    \begin{itemize}
        \item 20626 詹凱智:主要教學、製作講義
        \item 20520 林樺德:助教
    \end{itemize}

    \pagebreak

    \noindent
    學員:
    \begin{itemize}
        \item 10108 陳思彤
        \item 10113 王松平
        \item 10221 李證皓
        \item 10614 盧品潔
        \item 10627 陳紀昀
        \item 20531 蔡裕廷
        \item 20920 林士玹
        \item 21031 黃堉鈞
    \end{itemize}

    \noindent
    共 10 位

    \section*{陸、活動時間}
    111學年度下學期,每週五 16:30 至 17:30 ,段考週停止。\\詳細時間:2/17、3/3、3/10、3/31、4/7、4/14、4/21、5/19、5/26、6/2、6/9\\共 11 天。

    \pagebreak

    \section*{柒、課程規劃}
    由於部份同學可能沒有C++的基礎,在111學年度上學期末,會先提供網路上的C++教學資源以及基礎題單在寒假做練習。\\
    \noindent
    下學期開始每週除段考週會開始進度,暫定內容如下。
    \begin{enumerate}
        \item 課前介紹
        \item 基礎競賽概論
        \item 實做能力加強
        \item 枚舉
        \item 二分搜尋
        \item 貪心法
        \item 圖論
        \item 動態規劃
    \end{enumerate}

    \noindent
    由於本課程的教學週數較為緊湊,且課堂時間有限,因此大部分的練習時間將被安排在課後或是透過規劃彈性練習週進行。此外,每次課程均會另外錄影,以便未報名或是當天請假的同學觀看。

    \section*{捌、經費使用}
    預計使用經費:講義印刷、報名/競賽宣傳單,若有更多經費可以使用則可當作獎學金等。

    \section*{玖、預計成效}
    預計這一系列課程可以讓學員達到 \textbf{APCS 四級分}以上以及\textbf{北市能力競賽優等}的程度。
    此外,更能夠延續課程內容,為未來的學弟妹提供相同的資源。
\end{document}