% ============= setup ============= %
% ======== package ======== %
\documentclass[mathserif]{beamer}
\usepackage{xeCJK}
\usepackage{graphicx}
\usepackage{xcolor}
\usepackage{setspace}
\usepackage{newtxmath}

% ======== font ======== %
\setCJKmainfont{Taipei Sans TC Beta}
\setCJKsansfont{Taipei Sans TC Beta}
\AtBeginDocument{%
    \DeclareSymbolFont{pureletters}{OML}{cmm}{m}{it}%
    \SetSymbolFont{pureletters}{bold}{OML}{cmm}{b}{it}%
}
\hypersetup{
    colorlinks=true,
    linkcolor=black,
    urlcolor=blue
}

% ======== theme ======== %
\renewcommand{\baselinestretch}{1.25}
\usetheme{Madrid}
\usecolortheme{crane}
\setbeamertemplate{items}[circle]
\setbeamertemplate{section in toc}{\inserttocsectionnumber.~\inserttocsection}
\AtBeginSection[]{
    \begin{frame}
        \vfill
        \centering
        \begin{beamercolorbox}[sep=8pt,center,shadow=true,rounded=true]{title}
            \usebeamerfont{title}\insertsectionhead\par%
        \end{beamercolorbox}
        \vfill
    \end{frame}
}

% ======== data ======== %
\title{二分搜尋}
\author{temmie}
\date{}

% ============= setup ============= %

\begin{document}

\begin{frame}
    \titlepage
\end{frame}

\begin{frame}
    \tableofcontents
\end{frame}

\section{二分搜簡介}

\begin{frame}
    \frametitle{二分搜簡介}
    \begin{itemize}
        \item 讓我們試試看玩終極密碼 :D
        \item<2-> 事實上,我們就是在做\textbf{搜尋答案}這件事情
        \item<2-> 我們總共搜尋了多少次?這樣的時間複雜度是多少呢?
    \end{itemize}
\end{frame}

\begin{frame}
    \frametitle{二分搜方法}
    \begin{itemize}
        \item 既然答案有可能在任何地方,那就中間切一刀吧!
        \item 這樣一定是最好,因為兩邊的機率都會是一半
        \item 二分搜就是\textbf{將資料切半,看答案在哪一邊}
        \item 時間複雜度是 $O(log_2\ n)$ ,因為每次資料量都會減少一半
    \end{itemize}

\end{frame}

\section{尋找元素}

\begin{frame}
    \frametitle{例題}
    \begin{block}{搜尋例題}
        \href{https://codeforces.com/edu/course/2/lesson/6/1/practice/contest/283911/problem/A}{題目連結}
        給你 $n$ 個\textbf{排序後}的元素 $a_1, a_2 \cdots a_n$\\
        對於 $q$ 個查詢 $q_i$,輸出是否存在上面元素中
        \vspace{0.5cm}
        \begin{itemize}
            \item $1 \leq n, q \leq 10^5$
            \item $0 \leq \mid a_i, q_i \mid \leq 10^9$
        \end{itemize}
    \end{block}
\end{frame}

\begin{frame}
    \frametitle{搜尋法}
    \begin{itemize}
        \item 還記得我們第一堂課的找書嗎?
        \item 如果硬找的話,時間複雜度會是 $O(n \times q)$,會 TLE
        \item<2-> 
    \end{itemize}
\end{frame}

\section{搜尋數值}

\section{搜尋答案}

\end{document}