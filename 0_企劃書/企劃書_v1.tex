\documentclass[12pt]{article}
\usepackage{xeCJK}
\setCJKmainfont[]{Noto Serif CJK TC}
\usepackage[top=2cm, bottom=2cm, left=3cm, right=3cm]{geometry}
\usepackage{setspace}
\onehalfspacing
\usepackage{xcolor}
\usepackage{graphicx}

\begin{document}
    \begin{Huge}
        \noindent
        資訊檢定 × 競程讀書會企劃書
    \end{Huge}

    \section{目標}
    目標為對資訊檢測(APCS)、競賽程式有興趣且有python / C++基礎的同學打造更好的學習環境以及共學夥伴

    \section{內容}
    以每週兩次的教學進度,以及一次的心得分享讓同學們不僅可以在自身的技術上進步,也可以和其他同學交流,了解不同的解題思路

    \section{地點}
    暫定為電腦教室ㄧ(只要有投影幕或是白板的教室都可以)

    \section{參與人員}
    幹部:
    \begin{itemize}
        \item 20626 詹凱智:主要教學、製作講義、
        \item 助教:X
        \item 20617 延冠勲:美術設計、宣傳
    \end{itemize}

    \noindent
    學員為高一優先,其次為高二、三\\
    預計8 $\sim$ 15位

    \section{時間}
    111學年度下學期,每週ㄧ、二、四12:30至13:00,段考週停止

    \pagebreak

    \section{詳細實施方式}
    由於部份同學可能沒有C++的基礎,在111學年度上學期末,會先提供網路上的C++教學資源以及基礎題單在寒假做練習\\
    \noindent
    下學期開始每週除段考週會開始進度,暫定進度如下(週數待規劃)
    \begin{enumerate}
        \item 實做能力加強
        \item 資訊競賽概論、時間複雜度、心得分享示範
        \item 語法技巧、程式習慣
        \item 基礎資料結構
        \item 基礎區間問題
        \item 枚舉
        \item 二分搜尋
        \item 貪心法
        \item 動態規劃
    \end{enumerate}

    \noindent
    由於週數以及中午練習時間較為緊湊,大部分的練習時間會放在課後或是彈性插入練習週\\
    以及每次的課程都會另外錄影,讓沒報名或是當天請假的同學觀看

    \section{備註}
    預計使用經費:講義印刷、報名/競賽宣傳單,若有更多經費可以使用則可當作獎學金等
\end{document}