% ============= setup ============= %
% ======== package ======== %
\documentclass[mathserif]{beamer}
\usepackage{xeCJK}
\usepackage{graphicx}
\usepackage{xcolor}
\usepackage{setspace}
\usepackage{newtxmath}

% ======== font ======== %
\setCJKmainfont{Taipei Sans TC Beta}
\setCJKsansfont{Taipei Sans TC Beta}
\AtBeginDocument{%
    \DeclareSymbolFont{pureletters}{OML}{cmm}{m}{it}%
    \SetSymbolFont{pureletters}{bold}{OML}{cmm}{b}{it}%
}
\hypersetup{
    colorlinks=true,
    linkcolor=black,
    urlcolor=blue
}

% ======== theme ======== %
\renewcommand{\baselinestretch}{1.25}
\usetheme{Madrid}
\usecolortheme{crane}
\setbeamertemplate{items}[circle]
\setbeamertemplate{section in toc}{\inserttocsectionnumber.~\inserttocsection}
\AtBeginSection[]{
    \begin{frame}
        \vfill
        \centering
        \begin{beamercolorbox}[sep=8pt,center,shadow=true,rounded=true]{title}
            \usebeamerfont{title}\insertsectionhead\par%
        \end{beamercolorbox}
        \vfill
    \end{frame}
}

% ======== data ======== %
\title{資料結構}
\author{temmie}
\date{}

% ============= setup ============= %

\begin{document}

\begin{frame}
    \titlepage
\end{frame}

\begin{frame}
    \tableofcontents
\end{frame}

\begin{frame}{堆疊:Stack}
    \begin{itemize}
        \item 一種後進先出 (Last-In-First-Out, LIFO) 的線性資料結構
        \item 和疊盤子一樣,要放只能疊上去,要拿只能從最上面拿
    \end{itemize}
\end{frame}

\begin{frame}{堆疊:Stack}
    使用時機:
    \begin{itemize}
        \item 消除元素 (括號序列)
        \item 單調堆疊 (stack 裡的東西保持單調性)
        \item 四則運算解析
        \item 模擬遞迴
    \end{itemize}
\end{frame}

\begin{frame}{堆疊:Stack}
    STL 中現成的工具:
    \begin{itemize}
        \item 消除元素 (括號序列)
        \item 單調堆疊 (stack 裡的東西保持單調性)
        \item 四則運算解析
        \item 模擬遞迴
    \end{itemize}
\end{frame}


\end{document}