% ============= setup ============= %
% ======== package ======== %
\documentclass[mathserif]{beamer}
\usepackage{xeCJK}
\usepackage{graphicx}
\usepackage{xcolor}
\usepackage{setspace}
\usepackage{newtxmath}

% ======== font ======== %
\setCJKmainfont{Taipei Sans TC Beta}
\setCJKsansfont{Taipei Sans TC Beta}
\AtBeginDocument{%
    \DeclareSymbolFont{pureletters}{OML}{cmm}{m}{it}%
    \SetSymbolFont{pureletters}{bold}{OML}{cmm}{b}{it}%
}
\hypersetup{
    colorlinks=true,
    linkcolor=black,
    urlcolor=blue
}

% ======== theme ======== %
\renewcommand{\baselinestretch}{1.25}
\usetheme{Madrid}
\usecolortheme{crane}
\setbeamertemplate{items}[circle]
\setbeamertemplate{section in toc}{\inserttocsectionnumber.~\inserttocsection}
\AtBeginSection[]{
    \begin{frame}
        \vfill
        \centering
        \begin{beamercolorbox}[sep=8pt,center,shadow=true,rounded=true]{title}
            \usebeamerfont{title}\insertsectionhead\par%
        \end{beamercolorbox}
        \vfill
    \end{frame}
}

% ======== data ======== %
\title{資料結構}
\author{temmie}
\date{}

% ============= setup ============= %

\begin{document}

\begin{frame}
    \titlepage
\end{frame}

\begin{frame}
    \tableofcontents
\end{frame}

\section{Stack}

\begin{frame}{Stack}
    \begin{itemize}
        \item 一種後進先出 (Last-In-First-Out, LIFO) 的線性資料結構
        \item 和疊盤子一樣,要放只能疊上去,要拿只能從最上面拿
    \end{itemize}
\end{frame}

\begin{frame}{Stack}
    使用時機:
    \begin{itemize}
        \item 消除元素 (括號序列)
        \item 單調堆疊 (stack 裡的東西保持單調性)
        \item 四則運算解析
        \item 模擬遞迴
    \end{itemize}
\end{frame}

\begin{frame}{Stack}
    現成的工具:
    \begin{itemize}
        \item stack<{\color{red}type}> {\color{red}name}
        \begin{itemize}
            \item .push({\color{red}val})
            \item .pop()
            \item .top()
            \item .size()
            \item .empty()
        \end{itemize}
    \end{itemize}
\end{frame}

\section{Queue}

\begin{frame}{Queue}
    \begin{itemize}
        \item 一種先進先出 (First-In-First-Out, FIFO) 的線性資料結構
        \item 和排隊一樣,無法插隊,先到先走
    \end{itemize}
\end{frame}

\begin{frame}{Queue}
    使用時機:
    \begin{itemize}
        \item BFS
        \item 只需要動到頭尾
    \end{itemize}
\end{frame}

\begin{frame}{Queue}
    現成的工具:
    \begin{itemize}
        \item queue<{\color{red}type}> {\color{red}name}
        \begin{itemize}
            \item .push({\color{red}val})
            \item .pop()
            \item .front()
            \item .back()
            \item .size()
            \item .empty()
        \end{itemize}
    \end{itemize}
\end{frame}

\section{Deque}

\begin{frame}{Deque}
    \begin{itemize}
        \item 簡單來說就是 Stack + Queue
        \item \textbf{常數非常大},如非必要就不要使用
        \item 可以自己用陣列實作
    \end{itemize}
\end{frame}

\begin{frame}{Deque}
    現成的工具:
    \begin{itemize}
        \item deque<{\color{red}type}> {\color{red}name}
        \begin{itemize}
            \item .push\_front({\color{red}val})
            \item .push\_back({\color{red}val})
            \item .pop\_front()
            \item .pop\_back()
            \item .front()
            \item .back()
            \item .size()
            \item .empty()
            \item .clear()
            \item {\color{red}name}[{\color{red}i}]
        \end{itemize}
    \end{itemize}
\end{frame}

\section{Priority\_queue}

\begin{frame}{Priority\_queue}
    \begin{itemize}
        \item 一個可以自動排序的 Queue,但實際上是樹狀結構
        \item 會讓 Queue 由大到小(或是相反)排序
        \item 比起我們待會會講的 Set / Multiset 還要快
    \end{itemize}
\end{frame}

\begin{frame}{Priority\_queue}
    使用時機:
    \begin{itemize}
        \item 解決貪心問題
    \end{itemize}
\end{frame}

\begin{frame}{Priority\_queue}
    現成的工具:
    \begin{itemize}
        \item priority\_queue<{\color{red}type}>: 由大到小
        \item priority\_queue<{\color{red}type}, vector<{\color{red}type}>, greater<{\color{red}type}> >: 由小到大
        \item .push({\color{red}val})
        \item .pop()
        \item .top()
        \item .size()
        \item .empty()
    \end{itemize}
\end{frame}

\section{Pair}

\begin{frame}{Pair}
    \begin{itemize}
        \item 和我們之前學的 Struct 一樣
        \item 只有兩個空間,前面叫做 first 後面叫做 second
        \item 如果不想用 Struct 就可以用 Pair
    \end{itemize}
\end{frame}

\begin{frame}{Priority\_queue}
    現成的工具:
    \begin{itemize}
        \item pair<{\color{red}type1}, {\color{red}type2}>,兩個元素
        \item pair<{\color{red}type1}, pair<{\color{red}type2}, {\color{red}type3}> >,三個元素
        \item .first
        \item .second
    \end{itemize}
\end{frame}

\section{Set}

\begin{frame}{Set}
    \begin{itemize}
        \item 一種平衡二元樹,可以先忽略實作方法
        \item 使用時機非常多,不單單只是做為「集合」使用
    \end{itemize}
\end{frame}

\begin{frame}{Set}
    使用時機:
    \begin{itemize}
        \item 去重
        \item 動態排序(動態做二分搜能做的事)
        \item 快速知道前/後一個元素為何
    \end{itemize}
\end{frame}

\begin{frame}{Set}
    現成的工具:
    \begin{itemize}
        \item set<{\color{red}type}> {\color{red}name}
        \item .begin()
        \item .end()
        \item .find({\color{red}val}),return iterator
        \item .size()
        \item .empty()
    \end{itemize}
\end{frame}

\begin{frame}{Set}
    現成的工具:
    \begin{itemize}
        \item .insert({\color{red}val}),return pair<iterator, bool>
        \begin{itemize}
            \item $O(\log{n})$
            \item iterator 為插入的位置
            \item bool 為是否插入成功(如果已經存在就回傳 False)
        \end{itemize}
    \end{itemize}
\end{frame}

\begin{frame}{Set}
    現成的工具:
    \begin{itemize}
        \item .erase({\color{red}iterator}),return iterator
        \\.erase({\color{red}iterator\_left}, {\color{red}iterator\_right}),return iterator
        \\.erase({\color{red}val}),return bool
        \begin{itemize}
            \item $O(\log{n})$
            \item 傳入 iterator,則回傳下個元素的 iterator
            \item 傳入數值,則回傳有是否被刪除
        \end{itemize}
    \end{itemize}
\end{frame}

\begin{frame}{Set}
    現成的工具:
    \begin{itemize}
        \item upper\_bound / lower\_bound({\color{red}val}),return iterator
        \begin{itemize}
            \item $O(\log{n})$
            \item iterator 為找到的位置
        \end{itemize}
    \end{itemize}
\end{frame}

\begin{frame}{Unordered\_set}
    \begin{itemize}
        \item 和 set 差不多,但是不會有排序功能
        \item insert 和 erase 變成 $O(1)$
        \item 通常會用在一個叫做\textbf{雜湊}的技巧
    \end{itemize}
\end{frame}

\section{Multiset}
\begin{frame}{Multiset}
    \begin{itemize}
        \item 和 set 差不多,但沒有去重的功能
    \end{itemize}
\end{frame}

\begin{frame}{Multiset}
    現成的工具:
    \begin{itemize}
        \item .insert({\color{red}val}),return iterator
        \begin{itemize}
            \item $O(\log{n})$
            \item iterator 為插入的位置
        \end{itemize}
    \end{itemize}
\end{frame}

\begin{frame}{Multiset}
    現成的工具:
    \begin{itemize}
        \item .erase({\color{red}iterator}),return iterator
        \\.erase({\color{red}iterator\_left}, {\color{red}iterator\_right}),return iterator
        \\.erase({\color{red}val}),return bool
        \begin{itemize}
            \item $O(\log{n})$
            \item 傳入 iterator,則回傳下個元素的 iterator
            \item 傳入數值,則\textbf{所有和該數值相同的都會被刪除}
            \item 如果只想要刪除一個則使用 .erase(.find({\color{red}val}))
        \end{itemize}
    \end{itemize}
\end{frame}

\begin{frame}{Multiset}
    現成的工具:
    \begin{itemize}
        \item .count({\color{red}val}),return int
        \begin{itemize}
            \item $O($元素個數$)$
            \item 回傳元素個數,時間複雜度\textbf{與元素個數成正比},因此不該使用 Multiset 計算元素數量
        \end{itemize}
    \end{itemize}
\end{frame}

\section{Map}

\begin{frame}{Map}
    \begin{itemize}
        \item Map 有「映照」的意思,也就是將 key 對照成 value
        \item 同樣是以樹狀結構實作
        \item 同樣有自動排序
    \end{itemize}
\end{frame}

\begin{frame}{Map}
    使用時機:
    \begin{itemize}
        \item 沒辦法用陣列的索引值儲存
        \item 將資料做對應
    \end{itemize}
\end{frame}

\begin{frame}{Map}
    注意事項:
    \begin{itemize}
        \item Map 的 iterator 都是指向一個 pair,前者是 Key,後者是 Value
        \item 不要資料結構上癮,\textbf{如果索引值較小就應該用陣列}
        \item 在尋找元素之前,應該先確保該元素存在(用 .find 檢查),否則會因為該位置被初始化而導致時間複雜度爛掉
    \end{itemize}
\end{frame}

\begin{frame}{Map}
    現成的工具:
    \begin{itemize}
        \item map<{\color{red}type1}, {\color{red}type2}> {\color{red}name}
        \item .begin()
        \item .end()
        \item .find({\color{red}val}) return iterator
        \item .size()
        \item .empty()
        \item {\color{red}name} [{\color{red}key}]
    \end{itemize}
\end{frame}

\section{Bitset}

\begin{frame}{Bitset}
    \begin{itemize}
        \item 比起 bool[],是以精確的 bit 來實作,空間少上許多
        \item 做位元運算超快,甚至可以快到 $\frac{1}{64}$ 倍
    \end{itemize}
\end{frame}

\begin{frame}{Bitset}
    使用時機:
    \begin{itemize}
        \item 需要儲存 01 的資料
        \item 大量用到位元運算
        \item 壓常數
    \end{itemize}
\end{frame}

\begin{frame}{Bitset}
    現成的工具:
    \begin{itemize}
        \item bitset<{\color{red}固定大小}> {\color{red}name}
        \item .count()
        \item .size()
        \item .all()
        \item .any()
        \item .set()
        \item .reset()
        \item .to\_string()
        \item .to\_ulong()
        \item .to\_ullong()
        \item {\color{red}name} [{\color{red}key}]
    \end{itemize}
\end{frame}

\section{雜項}

\begin{frame}{雜項}
    以下是我沒講到的,它們出場率很低
    \begin{itemize}
        \item Tuple
        \item Unordered\_map
        \item List
    \end{itemize}
\end{frame}

\section{題目}

\begin{frame}{題目}
    \begin{itemize}
        \item \href{https://atcoder.jp/contests/agc005/tasks/agc005_a}{STring}
        \item \href{https://codeforces.com/contest/612/problem/C}{括號配對}
        \item \href{https://cses.fi/problemset/task/1645}{最近較小數字}
        \item \href{https://atcoder.jp/contests/abc247/tasks/abc247_d}{儲存站}
        \item \href{https://atcoder.jp/contests/abc237/tasks/abc237_d}{LR 排列}
        \item \href{https://cses.fi/problemset/task/1161/}{木棒組合}
        \item \href{https://codeforces.com/group/H0qY3QmnOW/contest/333897/problem/H}{藤原千花與字串}
        \item \href{https://atcoder.jp/contests/abc217/tasks/abc217_d}{木棒切割}
    \end{itemize}
\end{frame}

\end{document}